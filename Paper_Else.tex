%Terminal Command = xelatex -pdf FileName
%This is model of LaTeX for Chinese and Japanese
%KazukiAmakawa
%Pretreatment========================================================================================
\documentclass[UTF8]{ctexart}
\usepackage{graphicx}
\usepackage{multicol}
\usepackage{geometry}
\usepackage{lingmacros}
\usepackage{tree-dvips}
\usepackage{hyperref}
\usepackage{amsmath}
\usepackage{amssymb}
\usepackage{listings} 
\usepackage{amssymb}
\usepackage{verbatim}

\geometry{left=2cm,right=2cm,top=2.4cm,bottom=2.2cm}
\title{Title}
\author{\textbf{Kazuki Amakawa}}
\date{\today}

%Title================================================================================================
\begin{document}
\maketitle
\noindent \textbf{介绍}\\
紹介\\

\noindent \textbf{关键词:} \\[20ex]
\thispagestyle{empty}
\newpage
\tableofcontents
\thispagestyle{empty}
\newpage
%Main=================================================================================================
%Section 1=================================================================================================
\setcounter{page}{1}
\newpage
\section{紹介}
紹介の内容

%Section 2=================================================================================================
\newpage
\section{RWIEA算法数学原理}
\subsection{RWIEA算法数学原理 - 迭代概率法数学原理}
\subsubsection{随机游走分类}
随机游走分类,实际上是一种半监督学习,并不需要知道所有数据(结点)的标签,而只需要知道部分结点的标签,及结点之间的关系即可。\\

%Reference============================================================================================
\newpage
\medskip
\begin{thebibliography}{9}
\bibitem{Toboggan}
J. Fairfield, "Toboggan contrast enhancement for contrast segmentation," [1990] Proceedings. 10th International Conference on Pattern Recognition, Atlantic City, NJ, 1990, pp. 712-716 vol.1. doi: 10.1109/ICPR.1990.118200

\bibitem{MCMC}
Markov chain Monte Carlo - From Wikipedia, the free encyclopedia
\\\texttt{$https://en.wikipedia.org/wiki/Markov\_chain\_Monte\_Carlo$}


\end{thebibliography}

\noindent 代码:\texttt{$$}\\
\end{document}



